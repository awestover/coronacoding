\documentclass{article}[11pt]
\usepackage[subtle]{savetrees}
\usepackage[left=1in, right=1in, top=1in, bottom=1in]{geometry}

\usepackage{amsthm}
\usepackage{amssymb}
\usepackage{amsmath}
\usepackage{mathtools}
\usepackage{listings}

\usepackage{fancyhdr}
\pagestyle{fancy}
\lhead{Alek Westover}
\rhead{}
\usepackage{hyperref}

\DeclareMathOperator{\E}{\mathbb{E}}
\DeclareMathOperator{\Var}{\text{Var}}
\DeclareMathOperator{\img}{Img}
\DeclareMathOperator{\polylog}{\text{polylog}}
\DeclareMathOperator{\st}{\text{ such that }}
\newcommand{\norm}[1]{\left\lVert#1\right\rVert}
\newcommand{\interior}[1]{%
  {\kern0pt#1}^{\mathrm{o}}%
}
\newcommand{\mb}{\mathbf}
\newcommand{\x}{\mathbf{x}}
\newcommand{\y}{\mathbf{y}}
\newcommand{\z}{\mathbf{z}}
\renewcommand{\d}{\mathrm{d}} %straight d for integrals
\newcommand{\De}{\Delta}
\renewcommand{\Re}{\mathrm{Re}}
\renewcommand{\Im}{\mathrm{Im}}
\newcommand{\setof}[2]{\left\{ #1\; : \;#2 \right\}}
\newcommand{\set}[1]{\left\{ #1\right\}}
\newcommand{\R}{\mathbb{R}}
\newcommand{\C}{\mathbb{C}}
\newcommand{\Z}{\mathbb{Z}}
\newcommand{\contr}[0]{\[ \Rightarrow\!\Leftarrow \]}
\newcommand{\defeq}{\vcentcolon=}
\newcommand{\eqdef}{=\vcentcolon}

\newtheorem{fact}{Fact}
\newtheorem{definition}{Definition}
\newtheorem{remark}{Remark}
\newtheorem{proposition}{Proposition}
\newtheorem{lemma}{Lemma}
\newtheorem{corollary}{Corollary}
\newtheorem{theorem}{Theorem}
\usepackage{mdframed}
\newmdtheoremenv{q}{Question}

\author{Alek Westover and Max Vigneras}
\title{Corona Coding Club Day 1 intro}
\begin{document}
\maketitle

Thoughts from the author: 
The problems are sorted by difficulty within each section.
Some of these problems are pretty hard.
Work together! Ask Questions!

Pre-Lesson Homework: do the first 2 pythonprogramming.net tutorials.
Post-Lesson Homeowrk: do the next 2 tutorials. Do any of these problems that you thought were interesting.

\section{Easy Stuff}
Demo program: 
\begin{lstlisting}
  name = input()
  print("Hello "+ name)
\end{lstlisting}

\begin{q}
  Create a program that prints all even numbers up to your age.
  -From "Python for kids"
\end{q}

\begin{q}
  Create a program that asks someone for their name, adn then compliments them.
\end{q}

\begin{q}
  Create a number guessing game. ("Im thinking of a number in [1,100]" User says "high and low", computer guesses numbers)
\end{q}


\section{Number Theory Stuff}
Demo program: check if a number is a perfect square:
\begin{lstlisting}
  def checkIsSquare(n):
    i = 1
    while i*i <= n:
      if n == i*i:
        return True
      i+=1
    return False
\end{lstlisting}


\begin{q}
  Write a program to compute the $n$-th fibonacci number.
  Recall $f_0=f_1=1$, $f_n=f_{n-1}+ f_{n-2}$.

  Bonus: use linear algebra to do it in time $O(\log n)$
\end{q}

\begin{q}
  Write a program that makes an array with $x[i]$ indicating whether $i$ is prime or not.
\end{q}

\begin{q}
  Write a function that checks if a number is prime.
\end{q}

\begin{q}
  Write a program to compute the greatest common divisor of 
  two numbers.

  Hint: you can bash it, or do the euclidean algorithm
\end{q}


% \section{Sorting Stuff}

% \begin{q}
%   Implement quicksort.

%   Bonus: look up the smoothed striding algorithm http://parallelpartition.surge.sh/.

% \end{q}

\section{Encryption Stuff}
Demo: reverse cipher
\begin{lstlisting}
"".join([chr(26-(ord(x)%26)+ord('a')) for x in input_string])
\end{lstlisting}

\begin{q}
  Make a program that does a caesar cipher. That is, it takes in some english text, and cyclically shifts all the letters by some key, for example 1.
\end{q}

% \begin{q}
%   RSA
% \end{q}

\section{Signal Stuff}
Demo program: makes a graph of some noise.  
\begin{lstlisting}
  import matplotlib.pyplot as plt
  import numpy as np
  data = np.random.rand(100)
  plt.plot(data)
  plt.show()
\end{lstlisting}

\begin{q}
  Print a thousand random numbers.
\end{q}

\begin{q}
  Generate a random signal with 1000 data points. Plot it. Make a smoothed version of the signal.

  Hint: use moving average
\end{q}

\end{document}
